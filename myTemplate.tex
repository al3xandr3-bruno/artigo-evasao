%
% Template for RBIE papers in LaTeX
%

% The above language combination is for this template document only.
% You should use one of the following:
\documentclass[english, spanish, brazilian]{RBIEarticle} % for papers in portuguese
%\documentclass[brazilian, spanish, english]{RBIEarticle} % for papers in english
%\documentclass[brazilian, english, spanish]{RBIEarticle} % for papers in spanish

% Papers in Portuguese or Spanish may require the following lines:
\usepackage[utf8]{inputenc} % chooses UTF-8 as the main character set
\usepackage[T1]{fontenc} % for correct syllable separation in accented words
% Pacotes para citações/referências ABNT
%usepackage[alf]{abntex2cite} % citações autor-data
%\usepackage[num]{abntex2cite} % citações numéricas
\usepackage{amsmath}
\usepackage{float}

% The next two statements are needed for the example table in this document
% (i.e. you don't necessarily need them in your own paper)
\usepackage{colortbl}
\definecolor{gray}{gray}{.8}

% Citations and references (Biblatex)
% Citations and references (Biblatex)
\usepackage[style=abnt]{biblatex}
\usepackage{csquotes}
\addbibresource{references.bib}

% Here goes the paper main title
\title{Evasão no Ensino Médio Brasileiro: Uma Investigação do Impacto das Variáveis Institucionais, Acadêmicas e Socioeconômicas}

% If the manuscript is written in English, then this element must be removed.
\titleinenglish{Dropout Rates in Brazilian High Schools: An Investigation of the Impact of the Institutional, Academic and Socioeconomic Variables}

% If the manuscript is written in English, then this element must be removed.
\titleinspanish{Deserción escolar en la enseñanza secundaria brasileña: una investigación del impacto de las variables institucionales, académicas y socioeconómicas}

% Here goes the paper author information (repeat for two or more authors)
\author{%
	\parbox{3.8cm}{%
		Bruno Alexandre Dias da Silva\\
		Universidade de São Paulo\\
		ORCID: \href{https://orcid.org/0000-0000-0000-0000}{0000-0000-0000-0000}\\
		brunoalexdias20@usp.br
	}
        \hspace{0.3cm}
	\parbox{3.8cm}{%
		Lucas Gurgel do Amaral\\
		Universidade de São Paulo\\
		ORCID: \href{https://orcid.org/0000-0000-0000-0000}{0000-0000-0000-0000}\\
		lucasgurgel@usp.br
	}
        \hspace{0.3cm}
        \parbox{3.8cm}{%
		Rafael de França\\
		Universidade de São Paulo\\
		ORCID: \href{https://orcid.org/0000-0000-0000-0000}{0000-0000-0000-0000}\\
		rafaeldefranca@usp.br
	}
        \hspace{0.3cm}
	\parbox{3.9cm}{\raggedright%
		Richard Pereira do Nascimento\\
		Universidade de São Paulo\\
		ORCID: \href{https://orcid.org/0000-0000-0000-0000}{0000-0000-0000-0000}\\
		rcdwoods@usp.br
	}
}

\Submission{dd/Mmm/yyyy}
\First_round_notif{dd/Mmm/yyyy}
\New_version{dd/Mmm/yyyy}
\Second_round_notif{dd/Mmm/yyyy}
\Camera_ready{dd/Mmm/yyyy}
\Edition_review{dd/Mmm/yyyy}
\Available_online{dd/Mmm/yyyy}
\Published{dd/Mmm/yyyy}

% Here goes the page heading information
\heading{Gurgel et al.
}{RBIE v.VV – 2025}

% And finally here goes the citation information
\citeas{Last name, Initials., \ldots \& Last name, Initials.  (Year). Article title in the original language. Revista Brasileira de Informática na Educação, vol, pp-pp. https://doi.org/10.5753/rbie.yyyy.id}

\citeas{
SILVA, B. A. D.; GURGEL, L.; FRANÇA, R. D.; NASCIMENTO, R. P. do. Investigação do Impacto das Variáveis Institucionais, Acadêmicas e Socioeconômicas na Evasão no Ensino Médio Brasileiro. Revista Brasileira de Informática na Educação, vol , pp-pp, 2025.
}

%====================================================================
%\hyphenpenalty=10000
%\setcounter{page}{01}

\begin{document}
\maketitle

% If the manuscript is written in English, then this element must be removed.
\begin{otherlanguage}{brazilian}
\begin{abstract}
A evasão escolar no ensino brasileiro vem se mostrando como um grande desafio na formação educacional de crianças e jovens, principalmente nas camadas menos favorecidas da sociedade brasileira. Jovens que não concluem o ensino médio não conseguem se especializar em cursos superiores e, portanto, ocupam empregos de baixa remuneração, perpetuando um ciclo de pobreza e baixos indicadores socioeconômicos. Para isso, este trabalho tem como objetivo avaliar e comparar a relevância de variáveis institucionais, acadêmicas e socioeconômicas, a fim de investigar quais variáveis estão mais fortemente relacionadas com a evasão escolar no ensino médio brasileiro, por meio do modelo estatístico de Regressão Linear Múltipla avaliado por métricas como R². Da base de dados de Indicadores Educacionais e microdados do Censo Escolar do Instituto Nacional de Estudos e Pesquisas Educacionais Anísio Teixeira (INEP) serão extraídas variáveis que, de acordo com a literatura sobre evasão, melhor explicam o abandono escolar. Os resultados deste artigo sugerem que fatores de cunho acadêmico e socioeconômico são mais pertinentes ao fenômeno da evasão do que apenas as variáveis institucionais.
\keywords\ evasão escolar; variáveis acadêmicas; variáveis socioeconômicas; variáveis institucionais; INEP; ensino médio; modelagem estatística; regressão linear.
\end{abstract}
\end{otherlanguage}

\begin{otherlanguage}{english}
\begin{abstract}
School dropout in Brazilian education has proven to be a major challenge in the educational development of children and young people, especially in the less privileged segments of Brazilian society. Young people who do not complete high school are unable to specialize in higher education courses and, consequently, take low-paying jobs, perpetuating a cycle of poverty and low socioeconomic indicators. Therefore, this study aims to evaluate and compare the relevance of institutional, academic and socioeconomic variables in order to investigate which variables are most strongly associated with high school dropout in Brazil, using a Multiple Linear Regression statistical model evaluated through metrics such as R². Data from the Educational Indicators database and microdata from the School Census of the Anísio Teixeira National Institute for Educational Studies and Research (INEP) were used to extract variables that, according to the literature on dropout, best explain school abandonment. The results of this study suggest that academic and socioeconomic factors are more relevant to the dropout phenomenon than institutional variables alone.
\keywords\ school dropout; academic variables; socioeconomic variables; institutional variables; INEP ; high school; statistical modeling; linear regression.
\end{abstract}
\end{otherlanguage}

% If the manuscript is written in English, then this element must be removed.
\begin{otherlanguage}{spanish}
\begin{abstract}
La deserción escolar en la educación brasileña se ha convertido en un gran desafío para la formación educativa de niños y jóvenes, especialmente en los sectores menos favorecidos de la sociedad brasileña. Los jóvenes que no completan la educación secundaria no pueden especializarse en cursos de educación superior y, por lo tanto, ocupan empleos de baja remuneración, perpetuando un ciclo de pobreza y bajos indicadores socioeconómicos. Por ello, este estudio tiene como objetivo evaluar y comparar la relevancia de variables institucionales, académicas y socioeconómicas, con el fin de investigar qué variables están más fuertemente asociadas a la deserción escolar en la educación secundaria brasileña, mediante un modelo estadístico de Regresión Lineal Múltiple, evaluado por métricas como R². De la base de datos de Indicadores Educativos y microdatos del Censo Escolar del Instituto Nacional de Estudios e Investigaciones Educativas Anísio Teixeira (INEP) se extrajeron variables que, según la literatura sobre deserción, explican mejor el abandono escolar. Los resultados de este estudio sugieren que los factores académicos y socioeconómicos son más pertinentes al fenómeno de la deserción que las variables institucionales por sí solas.
\keywords\ deserción escolar; variables académicas; variables socioeconómicas; variables institucionales; INEP; educación secundaria; modelado estadístico; regresión lineal.
\end{abstract}
\end{otherlanguage}

\pagebreak

%====================================================================

\section{Introdução}

A evasão escolar constitui um dos principais desafios para a educação brasileira, representando não apenas a interrupção de trajetórias individuais, mas também a perpetuação de desigualdades sociais e o comprometimento do desenvolvimento econômico do país. No contexto do ensino médio, etapa final da educação básica, o fenômeno assume contornos particularmente preocupantes: segundo indicadores de fluxo escolar do Censo Escolar (INEP, 2023), o ensino médio apresenta taxa de abandono de aproximadamente 6\%, com disparidades significativas entre as redes de ensino. A rede pública concentra os maiores índices de evasão, enquanto a rede privada apresenta taxas substancialmente inferiores, evidenciando desigualdades estruturais que ultrapassam questões meramente pedagógicas.

Compreender os fatores associados à evasão exige reconhecer sua natureza multidimensional. A literatura nacional e internacional aponta para a convergência de elementos socioeconômicos — como baixa renda familiar, trabalho precoce e insegurança alimentar —, fatores acadêmicos — repetência, distorção idade-série e baixo desempenho — e variáveis institucionais — infraestrutura escolar, formação docente e tamanho de turmas (SILVA, 2016; ARAQUE; ROLDÁN; SALGUERO, 2009). Essa complexidade demanda abordagens analíticas capazes de quantificar a contribuição relativa de cada dimensão e orientar políticas públicas baseadas em evidências.

Dados do IBGE (2024) mostram que cerca de 8,7 milhões de jovens entre 14 e 29 anos abandonaram os estudos ou nunca frequentaram a escola, sendo que parcela significativa não havia concluído o ensino médio. As consequências desse abandono são duradouras: a renda média de trabalhadores com ensino médio completo é significativamente superior à de quem abandonou os estudos antes dessa etapa, perpetuando ciclos de vulnerabilidade e limitando oportunidades de mobilidade social. Em perspectiva comparada, o Brasil apresenta taxas de abandono escolar superiores à média de países da América Latina e muito distantes das registradas em nações com sistemas educacionais mais consolidados, evidenciando a necessidade urgente de políticas públicas de permanência e combate à evasão.

Apesar da relevância do tema, ainda persiste lacuna de estudos que integrem, em escala nacional, dados educacionais e socioeconômicos para análise sistemática dos determinantes da evasão. Grande parte da literatura brasileira concentra-se em recortes locais ou institucionais, com ênfase em modelos preditivos de risco individual voltados ao ensino superior. Investigações que articulem bases de dados oficiais — como os microdados do Censo Escolar, do Sistema de Avaliação da Educação Básica (SAEB) e da Pesquisa Nacional por Amostra de Domicílios (PNAD) — para analisar o fenômeno em nível municipal, identificando padrões territoriais e desigualdades regionais, permanecem escassas.

Nesse contexto, este trabalho propõe uma análise explicativa da evasão no ensino médio brasileiro a partir de modelagem estatística baseada em Regressão Linear Múltipla. O objetivo é investigar em que medida variáveis institucionais — como formação docente, infraestrutura e composição de turmas —, variáveis acadêmicas — taxa de repetência, distorção idade-série e desempenho em avaliações — e variáveis socioeconômicas — renda per capita, trabalho precoce e acesso a programas sociais — contribuem para explicar a variação das taxas de evasão entre municípios brasileiros.

A escolha da Regressão Linear Múltipla justifica-se por sua capacidade de quantificar o impacto individual de cada variável sobre a taxa de evasão, oferecendo coeficientes diretamente interpretáveis que facilitam a compreensão dos mecanismos subjacentes ao fenômeno e a aplicação dos resultados em políticas educacionais. Diferentemente de abordagens orientadas à predição de risco individual, este estudo privilegia a análise agregada em nível municipal, permitindo identificar contextos territoriais de maior vulnerabilidade e orientar estratégias de intervenção direcionadas.

Para tanto, foram integradas bases de dados do Instituto Nacional de Estudos e Pesquisas Educacionais Anísio Teixeira (INEP) — incluindo indicadores educacionais do Censo Escolar e microdados do SAEB — e da Pesquisa Nacional por Amostra de Domicílios (PNAD) do IBGE. Essa articulação possibilita examinar simultaneamente dimensões escolares e sociais, ampliando a compreensão sobre como fatores estruturais, pedagógicos e socioeconômicos interagem na determinação da evasão escolar no ensino médio brasileiro. Os resultados obtidos visam subsidiar gestores públicos, pesquisadores e educadores na formulação de políticas mais efetivas para a redução da evasão e promoção da equidade educacional.



\section{Fundamentos Teóricos}
A seção descreve a PNAD, INEP, o Censo Escolar, e SAEB. Além disso, descreve as técnicas estatísticas e computacionais empregadas na pesquisa.




\subsection{Regressão Linear Múltipla}
Entre as abordagens utilizadas para a análise de fatores que influenciam fenômenos complexos, destaca-se a Regressão Linear Múltipla, que permite investigar a relação entre uma variável dependente contínua e múltiplas variáveis independentes. Esse método possibilita compreender de forma quantitativa o impacto de diferentes fatores sobre o resultado observado, identificando quais variáveis apresentam maior influência e em que magnitude (MONTGOMERY; PECK; VINING, 2012).

O modelo de regressão linear múltipla, expresso por:

\vspace{0.5cm}
\begin{equation}
\large Y = \beta_0 + \beta_1X_1 + \beta_2X_2 + \beta_3X_3 + \cdots + \beta_iX_i + \epsilon 
\end{equation}
\vspace{0.5cm}

Em que $Y$ é a variável dependente a ser modelada, $X_i$ são as variáveis independentes, ou regressores, $\beta_i$ representa os coeficientes atrelados a cada variável independente e $\epsilon$ representa o erro aleatório. 

Com o modelo, e o auxílio da técnica dos mínimos quadrados para quantificar os coeficientes, é possível mensurar a significância das diferentes variáveis independentes, e o quanto sua variação impacta na variável dependente, de acordo com o contexto das amostras e das variáveis coletadas.


\subsection{A Pesquisa Nacional por Amostra de Domicílios (PNAD)}
A Pesquisa Nacional por Amostra de Domicílios (PNAD), realizada pelo Instituto Brasileiro de Geografia e Estatística (IBGE), constitui uma das principais fontes de dados socioeconômicos no Brasil. Seu objetivo é coletar informações abrangentes sobre características demográficas, educacionais, ocupacionais e de rendimento da população brasileira, por meio de entrevistas domiciliares aplicadas em amostras representativas em nível nacional, regional e estadual (IBGE, 2022).

A PNAD Contínua, em vigor desde 2012, aprimorou o levantamento ao adotar coleta trimestral, permitindo análises mais atualizadas e consistentes acerca da dinâmica social e econômica do país. Entre suas variáveis, destacam-se renda familiar per capita, inserção no mercado de trabalho, características do domicílio, composição familiar, escolaridade e acesso a programas sociais. Tais informações são de grande relevância para estudos sobre evasão escolar, pois possibilitam identificar relações entre vulnerabilidade socioeconômica e permanência na escola.

\subsection{Instituto Nacional de Estudos e Pesquisas Educacionais Anísio Teixeira (INEP)}
O Instituto Nacional de Estudos e Pesquisas Educacionais Anísio Teixeira (INEP), autarquia federal vinculada ao Ministério da Educação (MEC), tem como missão promover estudos, pesquisas e avaliações sobre o sistema educacional brasileiro. Fundado em 1937, o INEP é responsável pela produção e disseminação de informações estatísticas e avaliativas que subsidiam a formulação e o monitoramento de políticas públicas educacionais em âmbito nacional (INEP, 2023).

O Instituto realiza levantamentos censitários anuais em todas as etapas da educação básica e superior, além de coordenar avaliações de larga escala, como o Sistema de Avaliação da Educação Básica (SAEB) e o Exame Nacional do Ensino Médio (ENEM). Os dados produzidos pelo INEP constituem a principal fonte oficial de informações sobre matrícula, rendimento escolar, infraestrutura, profissionais da educação e fluxo escolar no país, sendo essenciais para a análise de fenômenos como evasão, repetência e distorção idade-série.

\subsubsection{Microdados do Censo Escolar}
O Censo Escolar, realizado anualmente pelo INEP em regime de colaboração com as secretarias estaduais e municipais de educação, constitui o principal instrumento de coleta de informações sobre a educação básica brasileira. Seu objetivo é reunir dados detalhados sobre estudantes, turmas, escolas e profissionais da educação em todas as redes de ensino – públicas e privadas –, abrangendo as etapas de educação infantil, ensino fundamental e ensino médio (INEP, 2023).

Os microdados do Censo Escolar são disponibilizados publicamente pelo INEP e permitem análises granulares sobre características individuais dos estudantes, como idade, sexo, raça/cor, situação de matrícula, modalidade de ensino e deficiências, além de informações sobre infraestrutura escolar, localização, dependência administrativa e recursos disponíveis. Esses dados são fundamentais para estudos sobre evasão escolar, pois possibilitam acompanhar longitudinalmente a trajetória estudantil, identificar fatores associados ao abandono e à reprovação, bem como analisar disparidades regionais e socioeconômicas que impactam o acesso e a permanência na escola.

\subsubsection{Sistema de Avaliação da Educação Básica (SAEB)}
O Sistema de Avaliação da Educação Básica (SAEB), também coordenado pelo INEP, constitui uma das principais avaliações externas e padronizadas da qualidade da educação no Brasil. Instituído em 1990, o SAEB tem como objetivo diagnosticar a educação básica brasileira e produzir indicadores sobre o desempenho dos estudantes em Língua Portuguesa e Matemática, além de coletar informações contextuais sobre condições de aprendizagem, perfil docente e clima escolar (INEP, 2023).

A partir de 2019, o SAEB passou a ser censitário para estudantes dos anos finais do ensino fundamental e do ensino médio, permitindo avaliações em nível de escola, município e estado. Os microdados do SAEB, disponibilizados junto aos do Censo Escolar, incluem resultados de proficiência, questionários contextuais aplicados a estudantes, professores e diretores, além de indicadores socioeconômicos e de infraestrutura. Tais informações são relevantes para estudos sobre evasão escolar, pois possibilitam relacionar desempenho acadêmico, contexto familiar e escolar com a permanência ou abandono dos estudantes, oferecendo subsídios para políticas educacionais mais direcionadas e eficazes.

\subsection{Multicolinearidade}

Multicolinearidade pode ser compreendida como uma relação linear entre duas ou mais variáveis dependentes. Conforme Paul (2006), quando há importância na investigação dos impactos dos regressores na variável dependente, a multicolinearidade pode ser um problema, visto que p-valores podem se mostrar equivocadamente elevados e em alguns casos pode interferir na interpretação dos coeficientes. Uma das maneiras de analisar se há multicolinearidade no conjunto de dados é calcular o fator de inflação de variância (\textit{VIF}), baseado em $R^2$, que indica o quanto uma variável independente é explicada pelos demais regressores para cada umas das $i$ variáveis independentes (MILOCA \& CONEJO, 2008).

O fator de inflação de variância do i-ésimo regressor ($F_i$)  é calculado por:

\vspace{0.5cm}
\begin{equation}
\large F_i = \frac{1}{1-R_i^2}
\end{equation}
\vspace{0.5cm}


Em que $R_i^2$ se refere ao $R^2$ da i-ésima variável independente.

Um \textit{VIF} maior que 10 indica que a multicolinearidade influencia fortemente o valor dos coeficientes do modelo, como proposto por Johnson e Wichern (1988; apud MILOCA \& CONEJO, 2008), e algumas medidas em relação ao conjunto de dados ou ao modelo devem ser tomadas com o objetivo de preservar a interpretação dos dados na modelagem estatística.

\subsection{Métricas de Desempenho}
\subsubsection{Coeficiente de Determinação ($R^2$)}
$R^2$ e $R^2$ ajustado permitem identificar quanto o modelo explica a variação da variável dependente que varia de 0 a 1 (sendo 1 o ajuste perfeito dos dados pelo modelo) e são calculados, respectivamente, por:

\vspace{0.5cm}
\begin{equation}
\large R^2 = 1 - \frac{\sum_i(y_i-\hat{y}_i)^2}{\sum_i(y_i-\bar{y})^2}
\end{equation}
\vspace{0.5cm}

\vspace{0.5cm}
\begin{equation}
\large R^2_{\text{aj}} = 1 - \left(1 - R^2\right)\frac{n - 1}{n - k - 1}
\end{equation}
\vspace{0.5cm}


Em que $y_i$ é o valor real, $\hat{y_i}$ é valor predito pelo modelo e $\bar{y}$ é a média dos valores da variável dependente. $n$ é o número de observações da amostra de dados e $k$ o número de variáveis excluindo o intercepto.

Diferente do $R^2$, o $R^2$ ajustado leva em consideração o número de amostras de um conjunto de dados e a quantidade de variáveis independentes, penalizando modelos com muitos regressores ou poucas amostras. 

\subsubsection{Estatística F}
A estatística F, conforme Sureiman et al (2020), avallia se o modelo concede uma explicação dos dados acima da explicação fornecida por um modelo de regressão sem variáveis de fato explicativas, indicando se o conjunto de variáveis contribui de forma significativa para explicar a variação da variável dependente. Quanto maior o valor de $F$, mais significativo se mostra o modelo, sendo este calculado por:

\vspace{0.5cm}
\begin{equation}
\large F = \frac{R^2(n-k-1)}{k(1 - R^2)}
\end{equation}
\vspace{0.5cm}

Onde $n$ é o número de observações da amostra de dados e $k$ o número de variáveis excluindo o intercepto.


\section{Trabalhos Relacionados}

A evasão escolar é um fenômeno complexo e multifatorial que tem sido amplamente estudado em diferentes níveis e contextos educacionais. Pesquisas recentes convergem quanto à importância de combinar variáveis institucionais e socioeconômicas para compreender o abandono e a permanência escolar, especialmente no ensino médio. O consenso entre os autores é que as decisões de evasão não se explicam apenas por características individuais, mas refletem também a estrutura e as condições das escolas, bem como fatores externos relacionados à renda e às oportunidades sociais(LOPES FILHO; SILVEIRA, 2021; BANAAG et al., 2024; SHIRASU; ARRAES, 2018; RESENDE; PETTERINI, 2022).

Nesse contexto, Lopes e Silveira (2021) analisaram dados administrativos da rede pública paulista com o objetivo de identificar estudantes em risco de evasão. Embora o trabalho utilize métodos de aprendizado de máquina, ele destaca a relevância de fatores contextuais ao nível da escola, como o tamanho das turmas, a razão aluno-professor e o desempenho médio da instituição. Os resultados mostram que escolas com turmas mais numerosas e menor desempenho tendem a apresentar taxas mais elevadas de abandono. Tais achados embasam a escolha, neste estudo, de variáveis como a \textit{média de alunos por turma} e o \textit{esforço docente}, que refletem a capacidade estrutural e a sobrecarga dos professores no ambiente escolar.

De modo complementar, a revisão sistemática conduzida por Banaag et al. (2024) identifica determinantes internos à escola como alguns dos principais preditores de evasão, destacando a baixa qualificação docente, as práticas pedagógicas inadequadas, a defasagem idade-série e a reprovação recorrente. Essas evidências sustentam a inclusão, neste trabalho, de variáveis que expressam o desempenho e a qualificação docente, como a \textit{taxa de reprovação} e a \textit{proporção de professores com formação superior}. Ambas representam dimensões críticas do processo de ensino-aprendizagem e ajudam a compreender a influência da qualidade da docência sobre a permanência dos alunos.

O estudo de Gusmão (2023) reforça esse papel da repetência no ensino médio brasileiro. Com base em séries históricas entre 2007 e 2019, o autor mostra que as taxas permanecem elevadas na rede pública, afetando principalmente estudantes de baixa renda e alimentando a distorção idade-série, a evasão e o baixo desempenho em avaliações nacionais, como o SAEB. Esses resultados são coerentes com a decisão deste trabalho de tratar a \textit{taxa de reprovação} como um dos principais mecanismos pelos quais o fracasso escolar se converte em abandono.

Em âmbito nacional, Shirasu e Arraes (2018) aplicaram modelos econométricos multiníveis ao ensino médio do Ceará, observando que repetência e atraso escolar elevam significativamente a probabilidade de abandono, enquanto políticas de transferência de renda, como o Bolsa Família, reduzem esse risco. O estudo reforça que as condições socioeconômicas e institucionais atuam de forma interdependente e que o enfrentamento da evasão requer abordagens que contemplem ambas as dimensões. Essa conclusão fundamenta a integração, neste artigo, de indicadores provenientes tanto do INEP quanto da PNAD, permitindo relacionar variáveis estruturais das escolas com características econômicas e sociais dos estudantes e de seus domicílios.

Na mesma direção, Resende e Petterini (2022), ao analisarem microdados administrativos do ensino médio catarinense, identificam correlação direta entre nível de renda familiar, histórico de repetência e maiores probabilidades de evasão. Os autores mostram que estudantes em situação socioeconômica mais vulnerável e com defasagem idade-série apresentam maior risco de abandono, o que reforça a importância de combinar indicadores de fluxo escolar com variáveis socioeconômicas de domicílio. Essa evidência dialoga com a estratégia adotada neste artigo de integrar informações do INEP e da PNAD em um modelo único.

A literatura também evidencia que a infraestrutura física da escola exerce papel relevante na permanência estudantil. Sousa et al. (2025) destacam que ambientes precários, ausência de laboratórios e de recursos tecnológicos, falta de manutenção e limitações no acesso à Internet reduzem o engajamento e aumentam as taxas de evasão. Nesse sentido, a inclusão de variáveis de \textit{infraestrutura escolar} — como presença de laboratório de ciências e informática, biblioteca, quadra esportiva e conectividade — permite avaliar o impacto das condições materiais sobre a probabilidade de abandono. Esses elementos complementam as dimensões institucionais e docentes, ampliando a capacidade explicativa do modelo.

Ainda que o presente estudo se concentre no ensino médio, pesquisas de outros níveis educacionais reforçam a relevância de abordagens quantitativas voltadas à interpretação de efeitos marginais entre variáveis institucionais. Silva et al. (2025), ao analisarem a evasão no ensino superior a distância, empregaram regressão linear múltipla para estimar o impacto de fatores como idade, gênero e tipo de escola de origem sobre as taxas de desistência. O uso da regressão permitiu mensurar a contribuição de cada variável e avaliar o desempenho do modelo por meio do \textit{erre ao quadrado} (R²), que indica o quanto da variação da evasão é explicada pelos regressores, e do \textit{erro quadrático médio} (MSE), que expressa a magnitude média dos erros de previsão. Essa abordagem reforça a adequação do método estatístico adotado neste trabalho, cujo foco recai sobre a interpretação e significância das variáveis independentes.

Em resumo, a literatura revisada demonstra que fatores institucionais — como o tamanho das turmas, o esforço docente e a infraestrutura — articulam-se a aspectos acadêmicos e socioeconômicos na explicação do abandono escolar no ensino médio. Em conjunto, os estudos analisados indicam que a evasão resulta da combinação de condições de oferta escolar, dinâmica do processo de ensino-aprendizagem e vulnerabilidades sociais que afetam, de forma desigual, os estudantes.

À luz desses estudos, o presente artigo contribui ao integrar, em escala nacional, indicadores institucionais e acadêmicos produzidos pelo INEP a variáveis socioeconômicas da PNAD Contínua, tomando o município como unidade de análise. Além disso, considera três recortes analíticos — municípios com padrões típicos de evasão, municípios com padrões atípicos e taxa de repetência por instituição —, o que permite comparar o peso relativo dos fatores escolares e socioeconômicos em diferentes configurações de risco. Ao privilegiar uma abordagem explicativa, centrada na interpretação dos coeficientes e na significância das variáveis, o estudo aprofunda a compreensão dos mecanismos associados à evasão e amplia o escopo da literatura, ao enfatizar a interpretação dos efeitos das variáveis em diferentes contextos de risco.




\section{Materiais e Métodos}

\subsection{Coleta e Preparação dos Dados}

Os dados utilizados neste estudo para análise estatística foram coletados a partir de bases dos  dados abertos do INEP, e correspondem ao ano de referência de 2017, uma vez que a análise não é conduzida em perspectiva temporal, e se concentra em um recorte pontual e típico (período pré-pandêmico).

A partir da literatura, apresentada na seção 3, as evidências apontam três grupos de variáveis determinantes da evasão: (a) \textit{variáveis institucionais e escolares}, uma vez que condições estruturais e organizacionais das escolas influenciam o engajamento e a permanência dos alunos; (b) \textit{indicadores de desempenho e qualificação docente explicam parte substancial do fenômeno}, pois refletem a qualidade do ensino e o acompanhamento pedagógico; e (c) \textit{condições socioeconômicas e materiais interagem com o contexto escolar}. 

Com base neste estudo da literatura foram criados dois conjuntos de dados. O primeiro que é chamado de Base de Evasão por Municípios neste trabalho e que é resultado da união dos microdados do censo escolar e dos indicadores educacionais do INEP a nível municipal, dada a disponibilização padronizada pelos bancos de dados consultados do instituto. O segundo é chamado Base de Taxa de Repetência por instituição que contém variáveis do SAEB do INEP a nível institucional, dada a acessibilidade a informações de instituições brasileiras nos registros do sistema.

\subsubsection{Criação da Base de Evasão por Município}
Para a seleção de variáveis da Base de Evasão por Município foram coletadas as bases de Esforço Docente, Média de Alunos Por Turma, Adequação da Formação Docente, Taxas de Distorção Idade-série e Taxas de Transição (Taxa de Repetência e Taxa de Evasão). Sendo comum entre as bases as variáveis: Região; UF; Código do Município; Nome do Município; Localização (urbana ou rural). Dentro de cada base foram apenas mantidas as colunas referentes ao ensino médio e selecionados apenas valores de Depedência Administrativa diferente de "Total", a fim de não causar interpretações indesejadas e manter apenas as dependências "Pública" e "Privada".


Do Censo Escolar foram extraídas as variáveis de presença de biblioteca nas instituições, presença de cozinha, presença de laboratório de ciências, presença de laboratório de informática, presença de quadra de esportes, presença de acessibilidade elevador, presença de professor psicólogo, presença de internet, presença de computadores \textit{desktop} para uso de alunos, presença de computadores portáteis para uso do aluno, presença de internet para acesso de alunos, presença de exame de seleção para ingresso na instituição, tipo de dependência da instituição ("Pública" ~ou "Privada"). Tais variáveis do Censo Escolar foram, portanto, agrupadas pelas colunas do tipo objeto (Código do Município, Dependência Administrativa, Regiao, UF e Nome do Municipio) em conjunto à função de agregação de soma e divisão da ocorrência do Código do Município, o que gerou taxas em decimal após o agrupamento com 9206 ocorrências.

Notada a irrelevância de algumas variáveis do tipo objeto para a análise estatística desta pesquisa, foram removidas as variáveis Região, UF e Localização de todas as bases do censo e dos indicadores educacionais, chegando-se, assim, nas variáveis selecionadas conforme as Tabelas \ref{tab:ind_inep_desc} e \ref{tab:var_censo_desc} que apresentam as variáveis selecionadas a partir dos indicadores educacionais e as variáveis do censo escolar para a Base de Evasão por Município, respectivamente.

\begin{table}[H]
\centering
\begin{tabular}{clp{7cm}}
\hline
\textbf{Índice} & \textbf{Variável} & \textbf{Descrição} \\ 
\hline
1  & TX\_REP\_TOTAL               & Taxa de repetência total por município  \\
2  & TX\_DI\_TOTAL                & Taxa de defasagem idade-série  \\
3  & NUM\_ALUNO\_TURMA\_TOTAL     & Número médio de alunos por turma total  \\
4  & DEP\_ADMINISTRATIVA\_Pública & Tipo de Dependência administrativa (Pública = 1; Privada = 0)\\
5  & TX\_IED\_N1                   & Taxa de índice de esforço docente nível 1      \\
6  & TX\_IED\_N2                   & Taxa de índice de esforço docente nível 2      \\
7  & TX\_IED\_N3                   & Taxa de índice de esforço docente nível 3      \\
8  & TX\_IED\_N4                   & Taxa de índice de esforço docente nível 4     \\
9  & TX\_IED\_N5                   & Taxa de índice de esforço docente nível 5      \\
10  & TX\_IED\_N6                   & Taxa de índice de esforço docente nível 6     \\
11 & Grupo 1                       & Taxa de grupo de docentes com formação superior de licenciatura na área de atuação      \\
12 & Grupo 2                       & Taxa de grupo de docentes com formação superior de bacharelado na área de atuação     \\
13 & Grupo 3                       & Taxa de grupo de docentes com formação superior de licenciatura em área diferente de atuação      \\
14 & Grupo 4                       & Taxa de grupo de docentes com formação superior não considera nas categorias anteriores     \\
15 & Grupo 5                       & Taxa de grupo de docentes sem formação superior   \\
\hline
\end{tabular}
\caption{Indicadores educacionais do INEP}
\label{tab:ind_inep_desc}
\end{table}


 \begin{table}[H]
\centering
\begin{tabular}{clp{7cm}}
\hline
\textbf{Índice} & \textbf{Variável} & \textbf{Descrição} \\ 
\hline
1  & IN\_ACESSIBILIDADE\_ELEVADOR  & Presença de acessibilidade a elevadores \\
2  & IN\_BIBLIOTECA                & Presença de biblioteca \\
3  & IN\_COMP\_PORTATIL\_ALUNO     & Presença de computadores portáteis para uso de alunos \\
4  & IN\_COZINHA                   & Presença de cozinha  \\
5  & IN\_DESKTOP\_ALUNO            & Presença de computadores de mesa para uso de alunos  \\
6 & IN\_EXAME\_SELECAO            & Presença de exame para ingresso  \\
7 & IN\_INTERNET                  & Presença de internet na instituição  \\
8 & IN\_INTERNET\_ALUNOS          & Presença de internet para uso de alunos  \\
9 & IN\_LABORATORIO\_CIENCIAS     & Presença de laboratório de ciências  \\
10 & IN\_LABORATORIO\_INFORMATICA  & Presença de laboratório de informática  \\
11 & IN\_PROF\_PSICOLOGO           & Presença de professores psicólogos  \\
12 & IN\_QUADRA\_ESPORTES          & Presença de quadra de esportes  \\
\hline
\end{tabular}
\caption{Variáveis do censo escolar do INEP}
\label{tab:var_censo_desc}
\end{table} 

Após a seleção, as diferentes bases foram mescladas de acordo com as variáveis em comum entre as bases (Código do Município, Dependência Administrativa, Nome do Município), o que resultou em 8235 amostras da Base de Evasão por Município antes de qualquer pré-processamento.




\subsubsection{Criação da Base de Taxa de Repetência por Instituição}
\label{sec:cr_base_saeb}
Na Base de Taxa de Repetência por Instituição foram selecionadas as variáveis: Identificador numérico da Escola; Ponto de Proficiência em Língua Portuguesa, Ponto de Proficiência em Matemática; se o aluno gosta de estudar matemática; se pais ou responsáveis incentivam o aluno a estudar; se o aluno gosta de estudar língua portuguesa; sexo do aluno; se o aluno trabalha fora de casa; se o aluno já reprovou; raça do aluno. Com um total de 1220572 amostras. A partir dessas variáveis foi executado um agrupamento de amostras por instituições de ensino pelo Identificador numérico da Escola em combinação com a função de agregação de soma de cada coluna, o que resultou em 18846 amostras da Base de Taxa de Repetência por Instituição. A Tabela \ref{tab:saeb_desc} mostra as variáveis selecionados do SAEB do INEP antes da etapa de pré-processamento dos dados.

\begin{table}[H]
\centering
\begin{tabular}{clp{7cm}}
\hline
\textbf{Índice} & \textbf{Variável} & \textbf{Descrição} \\
\hline
1 & PROFICIENCIA\_LP & Pontos de proficiencia em Língua Portuguesa \\
2 & PROFICIENCIA\_MT & Pontos de proficiência em Matemática  \\
3 & TX\_RESP\_Q001 & Sexo do aluno  \\
4 & TX\_RESP\_Q002 & Autodeclaração de raça de alunos  \\
5 & TX\_RESP\_Q038 & Alunos que trabalham fora  \\
6 & TX\_RESP\_Q044 & Alunos que gostam de estudar Língua Portuguesa  \\
7 & TX\_RESP\_Q052 & Alunos que gostam de estudar Matemática  \\
\hline
\end{tabular}
\caption{Variáveis do SAEB do INEP}
\label{tab:saeb_desc}
\end{table}

\subsection{Pré-processamento de Dados}

Após a coleta, ambos os conjuntos de dados mencionados na seção 4.2 foram submetidos a um processo estruturado de pré-processamento. 

Na Base de Taxa de Repetência por instituição, variáveis categóricas com resposta binária ("Sim" ~ou "Não") foram transformadas em variáveis binárias numéricas, em que "Sim" ~foi rotulado como 1 e "Não" ~como 0. Variáveis categóricas com diversas categorias possíveis, como exemplo a raça ("Branca", "Preta", "Amarela", "Parda", "Indígena"), foram transformadas em variáveis binárias (\textit{dummies}) de acordo com o número de classes possíveis da variável.

As taxas obtidas a partir da Base de Evasão por município, inicialmente no formato 0 a 100, foram transformadas para formato decimal (0 a 1) como:

\vspace{0.5cm}
\begin{equation}
\large T_k = \frac{t_k}{100}
\end{equation}
\vspace{0.5cm}

Em que $T_k$ representa a coluna modificada e $t_k$ representa a coluna da taxa original 

Células vazias ou nulas tiveram suas linhas removidas por completo da base de modelagem em ambos os conjuntos para integridade da análise estatística. A Base de Evasão por Município apresentou 6915 observações, e a Base de Taxa de Repetência por Instituição 18846, após a remoção completa de valores vazios.

\subsection{Análise Exploratória da Base de Evasão por Municípios e Análise de Multicolinearidade}

Foi realizada uma análise exploratória da taxa de evasão total por município para verificar se existem valores atípicos (\textit{outliers}).  Considerando essa análise (ver seção 5.1), a base de Evasão por Municípios foi divida em duas: (1) Base de dados de Evasão por Municipios típica com 6819 observações e (2) Base de dados de Evasão por Municípios atípica com 96 observações. 

Também foi realizada uma análise de multicolinearidade com base no conceito de Fator de Inflação de Variância (\textit{VIF}) com o objetivo de refinar a interpretação estatística dos dados e, por consequência, realizar a seleção de variáveis.

\subsection{Aplicação da Regressão Linear Múltipla na Base de Dados de Evasão por Município e da Base de Taxa de Repetencia por Instituição}

Por fim, foi aplicada Regressão Linear Múltipla sobre as bases de dados (1) e (2) da Base de Evasão por Município, com o intuito de investigar os impactos de cada variável na evasão do ensino médio brasileiro, bem como na Base de Taxa de Repetência por Instituição para avaliar os efeitos que cada variável tem na taxa de repetência das instituições de ensino.



\section{Resultados}

\subsection{Análise Exploratória da Base de Evasão por Municipios}

Na distribuição da variável taxa de evasão total (TX\_EV\_TOTAL), representada pela Figura 1, pode-se perceber que 25\% dos muncípios, levando em consideração instituições públicas e privadas, apresentam taxa de evasão inferior a 4,4\% do contingente de alunos matriculados no ensino médio regular. O segundo quartil, ou mediana, mostra que a medida central de distribuição está posicionada em torno de 7,8\%. Por outro lado, no terceiro quartil, ponto em que se encontram os 25\% maiores valores do conjunto, obteve-se uma evasão superior a 10,9\%.

Pelo intervalo interquartil foi calculado um limite superior de 20,65\%, ponto a partir do qual se posicionam os valores atípicos (\textit{outliers}) da distribuição da variável de taxa de evasão total por município. Deste modo, foram identificadas 96 ocorrências de valores acima do limite superior evidenciado pelo \textit{boxplot}, ainda na Figura 1.

Considerando essa análise, a base de Evasão por Municípios foi divida em duas: (1) Base de dados de Evasão por Municipios típica com 6819 observações e (2) Base de dados de Evasão por Municípios atípica com 96 
observações. Nesse sentido, busca-se uma interpretação consistente dos impactos das diferentes variáveis para cada subconjunto de dados, a fim de que não haja distorções em seus resultados.

\begin{figure}[H]
    \centering
    \includegraphics[width=0.56\linewidth]{dist_boxp_TX_EV.png}
    \caption{Boxplot da taxa de evasão total por município}
    \label{fig:placeholder}
\end{figure}

\subsection{Análise de Multicolinearidade na Base de Evasão por Município}


As Tabelas \ref{tab:vif_ind_inep_pos}, \ref{tab:vif_ind_censo_pre} e \ref{tab:vif_rep_pre} apresentam os resultados das análises de multicolinearidade do conjunto dos dados de evasão total por município.



Foi realizada a análise de VIF das variávies dos indicadores educacionais do INEP. Para as 
primeiras 4 variáveis da tabela, o $R^2 \geq 0,22$ e $VIF \geq 1,29$. Para o resto de variáveis da tabela $R^2=1$ e $VIF=\infty$. 

Os valores infinitos sugerem que há multicolinearidade perfeita entre os diferentes níveis das variáveis de taxa de esforço do docente e grupo de adequação da formação docente. Conforme estes resultados, foram eliminadas as variáveis de nível 1 e 2 da taxa de esforço docente (TX\_IED\_N1 e TX\_IED\_N2) e Grupo 1 da taxa de grupo de esforço docente, com o objetivo de mitigar a multicolinearidade no conjunto de dados.

A Tabela 2 mostra a  análise de VIF das variáveis dos indicadores educacionais do INEP após a eliminação das variáveis.

\begin{table}[H]
\centering
\begin{tabular}{ccc c}
\hline
\textbf{Índice} & \textbf{Variável} & \textbf{R\textsuperscript{2}} & \textbf{VIF} \\ 
\hline
0  & TX\_REP\_TOTAL               & 0.360672 & 1.564142 \\
1  & TX\_DI\_TOTAL                & 0.582384 & 2.394543 \\
2  & NUM\_ALUNO\_TURMA\_TOTAL     & 0.189437 & 1.233710 \\
3  & DEP\_ADMINISTRATIVA\_Pública & 0.530913 & 2.131803 \\
4  & TX\_IED\_N3                   & 0.710384 & 3.452852 \\
5  & TX\_IED\_N4                   & 0.677522 & 3.100985 \\
6  & TX\_IED\_N5                   & 0.555239 & 2.248396 \\
7  & TX\_IED\_N6                   & 0.363677 & 1.571528 \\
8  & Grupo 2                       & 0.094114 & 1.103892 \\
9  & Grupo 3                       & 0.212507 & 1.269852 \\
10 & Grupo 4                       & 0.108024 & 1.121106 \\
11 & Grupo 5                       & 0.140664 & 1.163690 \\
\hline
\end{tabular}
\caption{Análise de VIF das variáveis dos indicadores educacionais do INEP após eliminação de variáveis}
\label{tab:vif_ind_inep_pos}
\end{table}
 É possível notar na Tabela \ref{tab:vif_ind_inep_pos} que não há mais multicolinearidade entre as variáveis, o que diminui o ruído nas interpretações de coeficientes estatísticos do modelo.

As 12 variáveis da Tabela \ref{tab:vif_ind_inep_pos} foram unidas com as variáveis do Censo Escolar  do INEP (Tabela \ref{tab:var_censo_desc}). E a Tabela \ref{tab:vif_ind_censo_pre} mostra a análise de VIF desse novo conjunto de variáveis.

 \begin{table}[H]
\centering
\begin{tabular}{cccc}
\hline
\textbf{Índice} & \textbf{Variável} & \textbf{R\textsuperscript{2}} & \textbf{VIF} \\ 
\hline
0  & DEP\_ADMINISTRATIVA\_Pública & 0.628414 & 2.691170 \\
1  & Grupo 2                       & 0.113159 & 1.127598 \\
2  & Grupo 3                       & 0.251173 & 1.335423 \\
3  & Grupo 4                       & 0.121087 & 1.137769 \\
4  & Grupo 5                       & 0.183310 & 1.224454 \\
5  & IN\_ACESSIBILIDADE\_ELEVADOR  & $\text{NaN}$ & $\text{NaN}$ \\
6  & IN\_BIBLIOTECA                & 0.521304 & 2.089007 \\
7  & IN\_COMP\_PORTATIL\_ALUNO     & $\text{NaN}$ & $\text{NaN}$ \\
8  & IN\_COZINHA                   & 0.447879 & 1.811196 \\
9  & IN\_DESKTOP\_ALUNO            & $\text{NaN}$ & $\text{NaN}$ \\
10 & IN\_EXAME\_SELECAO            & $\text{NaN}$ & $\text{NaN}$ \\
11 & IN\_INTERNET                  & 0.692875 & 3.256002 \\
12 & IN\_INTERNET\_ALUNOS          & $\text{NaN}$ & $\text{NaN}$ \\
13 & IN\_LABORATORIO\_CIENCIAS     & 0.428596 & 1.750074 \\
14 & IN\_LABORATORIO\_INFORMATICA  & 0.521623 & 2.090401 \\
15 & IN\_PROF\_PSICOLOGO           & $\text{NaN}$ & $\text{NaN}$ \\
16 & IN\_QUADRA\_ESPORTES          & 0.566820 & 2.308507 \\
17 & NUM\_ALUNO\_TURMA\_TOTAL      & 0.221317 & 1.284219 \\
18 & TX\_DI\_TOTAL                 & 0.713851 & 3.494687 \\
19 & TX\_EV\_TOTAL                 & 0.514599 & 2.060153 \\
20 & TX\_IED\_N3                   & 0.712241 & 3.475127 \\
21 & TX\_IED\_N4                   & 0.679739 & 3.122457 \\
22 & TX\_IED\_N5                   & 0.558231 & 2.263629 \\
23 & TX\_IED\_N6                   & 0.375748 & 1.601918 \\
24 & TX\_REP\_TOTAL                & 0.383907 & 1.623132 \\
\hline
\end{tabular}
\caption{Análise de VIF das variáveis da Tabela \ref{tab:vif_ind_inep_pos} com as variáveis do censo escolar do INEP}
\label{tab:vif_ind_censo_pre}
\end{table} 

As 6 Variáveis que apresentam VIF igual a NaN da Tabela \ref{tab:vif_ind_censo_pre} foram removidas do conjunto de dados para manter a integridade da base.

Em síntese, inicialmente foram consideradas 25 variáveis para análise estatística dos impactos na taxa de evasão total por município típica e atípica. Após análise de multicolinearidade, com a remoção de variáveis com valor alto de VIF, obteve-se o conjunto preliminar de 19 variáveis para serem utilizadas na Regressão Linear Múltipla.

\subsection{Análise de Multicolinearidade na Base de Repetência por Instituição}

A Tabela \ref{tab:vif_rep_pre} apresenta a análise de multicolinearidade das variáveis selecionadas da base do SAEB.

\begin{table}[H]
\centering
\begin{tabular}{cccc}
\hline
\textbf{Índice} & \textbf{Variável} & $\mathbf{R^2}$ & \textbf{VIF} \\
\hline
0 & PROFICIENCIA\_LP & 0.875215 & 8.013791 \\
1 & PROFICIENCIA\_MT & 0.878331 & 8.219017 \\
2 & TX\_RESP\_Q001 & 0.071145 & 1.076594 \\
3 & TX\_RESP\_Q002\_Amarelo & 0.391791 & 1.644171 \\
4 & TX\_RESP\_Q002\_Branca & 0.960491 & 25.310711 \\
5 & TX\_RESP\_Q002\_Indígena & 0.659326 & 2.935356 \\
6 & TX\_RESP\_Q002\_Parda & 0.941405 & 17.066386 \\
7 & TX\_RESP\_Q002\_Preta & 0.824584 & 5.700737 \\
8 & TX\_RESP\_Q027 & 0.030971 & 1.031961 \\
9 & TX\_RESP\_Q038 & 0.203697 & 1.255804 \\
10 & TX\_RESP\_Q041 & 0.295520 & 1.419486 \\
11 & TX\_RESP\_Q044 & 0.199099 & 1.248594 \\
12 & TX\_RESP\_Q052 & 0.095574 & 1.105673 \\
\hline
\end{tabular}
\caption{Análise de VIF das variáveis do SAEB}
\label{tab:vif_rep_pre}
\end{table}

Pode-se notar um alto VIF nas variáveis de ponto de proficiência em matemática e língua portuguesa, assim como nas variáveis de raça. Em busca de reduzir a distorção na interpretação estatística foram removidas as variáveis de proficiência em língua portuguesa e as taxas de alunos autodeclarados brancos e amarelos. 

Sumariamente foram selecionadas 7 variáveis (ver Seção \ref{sec:cr_base_saeb}), que após a etapa de pré-processamento tornaram-se 13 variáveis para análise estatística dos fatores associados à taxa de repetência total por instituições de ensino, conforme a Tabela \ref{tab:vif_rep_pre}. Conduzida uma análise de multicolinearidade e a remoção de 3 variáveis que apresentaram alto valor de VIF (TX\_RESP\_Q002\_Branca; PROFICIENCIA\_LP; TX\_RESP\_Q002\_Amarelo), obteve-se o conjunto final de 9 variáveis para serem utilizadas na Regressão Linear Múltipla.

\subsection{Análise Exploratória da Taxa de Repetência e Taxa de Distorção Idade-série}
Gusmão (2023) ressalta com base em outros autores que o percentual de repetência está intimamente ligado à distorção idade-série e, principalmente, à evasão escolar no ensino médio brasileiro. Diante desse contexto foi feita uma análise das distribuições histogramas das variáveis TX\_REP\_TOTAL e TX\_DI\_TOTAL, que representam as taxas de repetência e distorção-idade respectivamente, nas Figuras 2 e 3.

\vspace{0.5cm}
\begin{figure}[H]
    \centering
    \includegraphics[width=0.56\linewidth]{hist_tx_rep.png}
    \caption{Histograma da variável do percentual de repetência por município}
    \label{fig:placeholder}
\end{figure}

\begin{figure}[H]
    \centering
    \includegraphics[width=0.56\linewidth]{hist_tx_di.png}
    \caption{Histograma da variável do percentual de distorção idade-série por município}
    \label{fig:placeholder}
\end{figure}

Na Figura 2, pode-se observar um padrão de unimodalidade esperado no percentual de repetência, com um pico entre 0 e 0,1. Contudo, é possível identificar um comportamento bimodal na variável de taxa distorção-idade (Figura 3), o que sugere a presença de dois subgrupos dentro da amostra de tal percentual neste trabalho. Para tanto, a fim de analisar estatisticamente a diferença entre os subgrupos, a variável TX\_DI\_TOTAL foi transformada em variável binária de tal modo que valores maiores ou iguais a 0,2 foram rotulados como 1, e, por conseguinte, valores abaixo de 0,2 foram rotulados como 0. Diante desse cenário, a variável foi renomeada como TX\_DI\_GRUPO\_1, fazendo referência a sua forma binária.



\subsection{Resultados da Regressão Linear Múltipla na Base de dados de Evasão por Municipios Típica}

A Tabela \ref{tab:reg_tipicos_tx_ev} apresenta os resultados da aplicação de Regressão Linear Múltipla na Base de dados de Evasão por Municipios Típica.

\begin{table}[H]
\centering
\caption{Resultados da Regressão Linear para Valores Típicos para TX\_EV\_TOTAL}
\label{tab:reg_tipicos_tx_ev}
\begin{tabular}{lcccc}
\hline
\textbf{Variável} & \textbf{Coef.} & \textbf{Erro-padrão} & \textbf{t} & \textbf{p-valor} \\
\hline
const & -0,0003 & 0,005 & -0,071 & 0,944 \\
DEP\_ADMINISTRATIVA\_Pública & 0,0400 & 0,002 & 24,773 & 0,000 \\
Grupo 2 & 0,0308 & 0,008 & 3,760 & 0,000 \\
Grupo 3 & 0,0003 & 0,003 & 0,085 & 0,932 \\
Grupo 4 & 0,0094 & 0,005 & 1,854 & 0,064 \\
Grupo 5 & 0,0159 & 0,004 & 3,867 & 0,000 \\
IN\_BIBLIOTECA & 0,00007014 & 0,002 & 0,032 & 0,974 \\
IN\_COZINHA & 0,0070 & 0,002 & 2,838 & 0,005 \\
IN\_INTERNET & 0,0014 & 0,002 & 0,581 & 0,561 \\
IN\_LABORATORIO\_CIENCIAS & -0,00004531 & 0,004 & -0,012 & 0,990 \\
IN\_LABORATORIO\_INFORMATICA & -0,0046 & 0,003 & -1,823 & 0,068 \\
IN\_QUADRA\_ESPORTES & -0,0019 & 0,003 & -0,747 & 0,455 \\
NUM\_ALUNO\_TURMA\_TOTAL & 0,0002 & 0,0000631 & 3,623 & 0,000 \\
TX\_DI\_GRUPO\_1 & 0,0294 & 0,001 & 25,130 & 0,000 \\
TX\_IED\_N3 & -0,0018 & 0,005 & -0,350 & 0,726 \\
TX\_IED\_N4 & 0,0022 & 0,005 & 0,491 & 0,624 \\
TX\_IED\_N5 & 0,0077 & 0,005 & 1,510 & 0,131 \\
TX\_IED\_N6 & 0,0240 & 0,006 & 3,813 & 0,000 \\
TX\_REP\_TOTAL & 0,1442 & 0,008 & 17,835 & 0,000 \\
\hline
\multicolumn{5}{l}{\textbf{R\textsuperscript{2}} = 0,511} \\
\multicolumn{5}{l}{\textbf{R\textsuperscript{2} ajustado} = 0,509} \\
\multicolumn{5}{l}{\textbf{Estatística F} = 394,4} \\
\multicolumn{5}{l}{\textbf{Número de observações} = 6819} \\
\hline
\end{tabular}
\end{table}


O modelo da Tabela \ref{tab:reg_tipicos_tx_ev} apresenta um $R^2=0,511$, explicando $51\%$ da variação dos dados, e uma $Prob(F)=0$, que indica que o modelo é significativo para explicar a variável dependente.

Os resultados mostram que instituições públicas apresentam um aumento de 4 pontos percentuais em comparação à rede privada de ensino no abandono contínuo das atividades escolares.
O percentual de grupo de docentes com formação na mesma área em que atuam (Grupo 2) e o percentual de grupo de docentes sem formação superior (Grupo 5), estão associados a um aumento de 0,030 e 0,015 pontos percentuais, respectivamente, na evasão, considerando um aumento de 1 ponto percentual em cada variável exclusivamente e mantendo outras variáveis constantes.

A taxa da presença de cozinhas em escolas (IN\_COZINHA) se mostra signficativa, mas sua variação representa um modesto aumento de 0,007 pontos percentuais na evasão. O número médio de alunos por turma (NUM\_ALUNO\_TURMA\_TOTAL) apresenta um baixo coeficiente de 0,0002 pontos percentuais. 

Ademais os resultados apontam que municípios com percentual de distorção idade-série superior a 20\% apresentam um aumento de 2,94 pontos percentuais em comparação àqueles com taxa menor que o limiar dos subgrupos.

A taxa de docentes que têm mais de 400 alunos e atua nos três turnos, em duas ou três escolas e em duas ou três etapas (TX\_IED\_N6) tem variação associada a um aumento de 0,02 pontos percentuais na evasão.

Por fim, a regressão aponta que uma variação de 1 ponto percentual na taxa de repetência (TX\_REP\_TOTAL) está associada a um aumento de 0,1442 pontos percentuais na taxa de evasão, o maior impacto na evasão neste trabalho.

Tais resultados corroboram o trabalho de Gusmão (2023) que salienta a taxa de repetência, principalmente nas escolas públicas, como principal determinante da evasão e da distorção idade-série no ensino secundário brasileiro. 

As demais variáveis não citadas, como a maioria de infraestrutura, grupos de adequação e taxas de esforço do docente, no contexto desta pesquisa, não se mostraram signifcativas.

\subsection{Resultados da Regressão Linear Múltipla na Base de dados de Evasão por Municipios Atípica}

Para valores atípicos,observou-se que apenas 4 das 18 variáveis, apresentaram significância estatística para um $p<0,05$. Com isso, a fim de diminuir um possível ruído e dificuldade de interpretação dos resultados, foi conduzida uma regressão linear apenas com as variáveis estatiscamente válidas.
 
A Tabela \ref{tab:reg_atipicos_tx_ev} apresenta os resultados da aplicação de Regressão Linear Múltipla na Base de dados de Evasão por Municipios Atípica.

\begin{table}[H]
\centering
\caption{Resultados da Regressão Linear para Valores Atípicos
\label{tab:reg_atipicos_tx_ev}
\textit{outliers} para TX\_EV\_TOTAL}
\begin{tabular}{lcccc}
\hline
\textbf{Variável} & \textbf{Coef.} & \textbf{Erro-padrão} & \textbf{t} & \textbf{p-valor} \\
\hline
const & 0,3079 & 0,037 & 8,248 & 0,000 \\
DEP\_ADMINISTRATIVA\_Pública & -0,1781 & 0,027 & -6,690 & 0,000 \\
IN\_LABORATORIO\_INFORMATICA & 0,0569 & 0,027 & 2,086 & 0,040 \\
TX\_DI\_GRUPO\_1 & 0,1282 & 0,044 & 2,941 & 0,004 \\
TX\_REP\_TOTAL & -0,2338 & 0,084 & -2,790 & 0,006 \\
\hline
\multicolumn{5}{l}{\textbf{R\textsuperscript{2}} = 0,445} \\
\multicolumn{5}{l}{\textbf{R\textsuperscript{2} ajustado} = 0,420} \\
\multicolumn{5}{l}{\textbf{Estatística F} = 18,21} \\
\multicolumn{5}{l}{\textbf{Número de observações} = 96} \\
\hline
\end{tabular}
\end{table}


Diferente do modelo dos valores típicos da Tabela \ref{tab:reg_tipicos_tx_ev}, instituições públicas têm uma diminuição de 17,81 pontos percentuais na evasão em comparação a instituições privadas que oferecem o ensino médio regular. 
A presença de laboratório de informática (IN\_LABORATORIO\_INFORMATICA) está associada ao aumento da evasão.

Nos valores atípicos, os municípios com percentual de distorção idade-série superior a 20\% estão associados a um aumento de 12,82 pontos percentuais na evasão em comparação com aqueles que pertencem ao subgrupo de municípios com percentual inferior a 20\%.

Por último, a variação na taxa de repetência está associada a uma diminuição da evasão em 0,23 pontos percentuais. Tais valores negativos, antes positivos e associados a um aumento da evasão, não puderam ser devidamente explicados no escopo desta pesquisa. Contudo é possível levantar a hipótese de que nos municípios que apresentam taxas de evasão atípicas xos valores negativos podem estar relacionados a fatores de vulnerabilidade alimentar, como exemplo, que não é abordada neste trabalho.

\subsection{Resultados da Regressão Linear Múltipla na Base de dados de Taxa de Repetencia por Instituição}

Diante desse cenário, decidiu-se utilizar a base do Sistema de Avaliação do Educação básica (SAEB) do INEP, para investigar quais variáveis são determinantes da taxa de repetência e, de forma implícita, da taxa de defasagem idade-série, dado que a repetência leva à defasagem em relação à série cursada.

A Tabela \ref{tab:regressao_tx_resp_q041} apresenta os resultados da aplicação de Regressão Linear Múltipla na Base de dados de Taxa de Repetência por Instituição.

\begin{table}[htbp]
\centering
\caption{Resultados da regressão linear para a taxa de repetência por instituição}
\label{tab:regressao_tx_resp_q041}
\small
\begin{tabular}{lrrrrr}
\hline
\textbf{Variável} & \textbf{Coeficiente} & \textbf{Erro padrão} & \textbf{t} & \textbf{p$>$|t|} & \textbf{[0,025 ; 0,975]} \\
\hline
const & 0,6980 & 0,039 & 17,716 & 0,000 & [0,621 ; 0,775] \\
PROFICIENCIA\_MT & -0,1567 & 0,003 & -61,733 & 0,000 & [-0,162 ; -0,152] \\
TX\_RESP\_Q001 & 0,0841 & 0,010 & 8,203 & 0,000 & [0,064 ; 0,104] \\
TX\_RESP\_Q002\_Indígena & 0,1083 & 0,018 & 5,879 & 0,000 & [0,072 ; 0,144] \\
TX\_RESP\_Q002\_Parda & 0,0474 & 0,007 & 6,716 & 0,000 & [0,034 ; 0,061] \\
TX\_RESP\_Q002\_Preta & 0,1422 & 0,012 & 11,985 & 0,000 & [0,119 ; 0,165] \\
TX\_RESP\_Q027 & -0,4332 & 0,039 & -11,121 & 0,000 & [-0,510 ; -0,357] \\
TX\_RESP\_Q038 & 0,0595 & 0,007 & 8,035 & 0,000 & [0,045 ; 0,074] \\
TX\_RESP\_Q044 & -0,1006 & 0,010 & -10,433 & 0,000 & [-0,120 ; -0,082] \\
TX\_RESP\_Q052 & 0,0965 & 0,009 & 11,240 & 0,000 & [0,080 ; 0,113] \\
\hline
\multicolumn{5}{l}{\textbf{R\textsuperscript{2}} = 0,272} \\
\multicolumn{5}{l}{\textbf{R\textsuperscript{2} ajustado} = 0,272} \\
\multicolumn{5}{l}{\textbf{Estatística F} = 782,4} \\
\multicolumn{5}{l}{\textbf{Número de observações} = 18846} \\
\hline
\end{tabular}
\end{table}

O modelo se mostra estatisticamente significativo dada $Prob(F)=0$, e explica 27,2\% da variação dos dados coletados.

Pode-se observar que das variáveis coletadas todas são significativas com $p<0,001$, e se destacam as variáveis TX\_RESP\_Q027, taxa de alunos incentivados pelos pais aos estudos, com um alto coeficiente de $\beta_i=-0,4332$, PROFICIENCIA\_MT ($\beta_i=-0,1567$), que mede a proficiência de matemática média da instituição; TX\_RESP\_Q002\_Preta ($\beta_i=0,1422$), que refere-se à taxa de alunos autodeclarados pretos na instituição de ensino; TX\_RESP\_Q002\_Indígena ($\beta_i=0,1083$), que faz jus à taxa de alunos autodeclarados indígenas. Tais variáveis possuem os maiores coeficientes do conjunto de regressores. 

De acordo com os resultados, o aumento de 1 ponto percentual na taxa de alunos incentivados ao estudo está associado à diminuição de 0,4332 pontos percentuais na taxa de repetência. De modo semelhante, o aumento de 1 ponto percentual na proficiência em matemática tem associação com a diminuição em 0,1567 pontos percentuais da variável dependente.

Por outro lado, o aumento de 1 ponto percentual no grupo de alunos autodeclarados pretos nas instituições tende a elevar a taxa de repetência em 0,1422. E o mesmo pode ser observado para o grupo de alunos indígenas, cujo aumento de 1 ponto percentual implica em $+0.1083$ pontos percentuais na taxa de repetência. Ademais, o aumento de 1 ponto percentual na taxa de alunos que trabalham (TX\_RESP\_Q038) está associado a $+0,0595$ pontos percentuais na taxa de repetência.

A variável TX\_RESP\_Q044, que faz referência à taxa de alunos que gostam de estudar língua portuguesa, também mostra que com o aumento de 1 ponto percentual, a taxa de repetência tende a $-0,1006$ pontos percentuais. E curiosamente, o aumento de 1 ponto percentual na taxa de alunos que demonstram interesse por matemática (TX\_RESP\_Q052) está associado a um aumento de aproximadamente 0,09 pontos percentuais na taxa de repetência.

\subsection{Relação entre Evasão, Repetência e Defasagem Idade-série}

A correlação entre evasão, repetência e defasagem idade-série é corroborada pela cenário do ensino médio brasileiro e a literatura, assim como evidenciado por Gusmão (2023). De acordo com dados da PNAD, pesquisa nacional realizada pelo IBGE, coletados nesta pesquisa para análise complementar, a taxa de evasão entre pretos em idade escolar adequada ao ensino secundário é de 22,87\%, entre pardos 22,26\%, e entre indígenas equivalente a 25,64\% (com base na lógica de cálculo de evasão deste trabalho). Brancos apresentam o menor número com cerca de 16,16\%. 

No índice de repetência, o grupo de pretos e pardos é cerca de 2 a 4 pontos percentuais acima do percentual de brancos que se mantêm na mesma série em todo o Brasil, de acordo com dados do INEP. Ainda segundo a plataforma inepdata, a taxa de jovens pretos e pardos que estão defasados no quesito idade-série, chega a ser de 10 pontos percentuais a mais em comparação com jovens brancos.

A evasão e dificuldade escolar de tais grupos fragilizados também são refletidos em suas condições socioeconômicas e o inverso também é verdadeiro. Jovens pretos em idade escolar que já estão fora do ensino médio devido à conclusão ou evasão apresentam renda média mensal de 790 reais, jovens pardos 780 reais, jovens indígenas 727 reais, e jovens brancos 1230 reais em média, segundo dados da PNAD no contexto desta pesquisa.

Ante o exposto, os resultados reforçam que, em alguma medida, fatores de infraestrutura e institucionais não são significativos para explicar a evasão e o índice de repetência, tal como fatores acadêmicos e sociais, como a raça e o incentivo doméstico aos estudos e manutenção de presença na instituição de ensino.


\pagebreak
\section{Conclusão}
Os resultados parciais e a análise estatística apontam que a evasão escolar no ensino médio brasileiro constitui um fenômeno fortemente associado a fatores acadêmicos e socioeconômicos, enquanto variáveis institucionais apresentam impacto relativamente pequeno na explicação do fenômeno. 

Com destaque para aspectos acadêmicos, a regressão linear múltipla realizada com dados do INEP indica que a taxa de defasagem idade-série e a taxa de repetência são os principais determinantes da taxa de abandono escolar, com coeficientes elevados e significativos, sobretudo nos municípios com padrões típicos de evasão. A análise complementar utilizando a base do SAEB reforçou esse entendimento, mostrando que variáveis relacionadas à proficiência em matemática e incentivo doméstico aos estudos possuem efeito relevante na repetência escolar, a qual, por sua vez, tende a levar à defasagem escolar e por fim ao abandono. 

Analisando elementos de cunho socioeconômico, nota-se que características demográficas, como autodeclaração racial, repercutem significativamente nas variáveis acadêmicas. Em particular, alunos pretos e indígenas apresentam maior probabilidade de repetência e defasagem idade-série, corroborando os dados da PNAD sobre evasão e desigualdades socioeconômicas, como rendimento domiciliar per capita inferior em comparação a alunos brancos.

Como limitação, destaca-se o foco da análise em um único ano de referência, o que abre caminho para investigações futuras que considerem os efeitos da pandemia e das recentes reformas no ensino médio.

Em síntese, este estudo evidencia que fatores acadêmicos e socioeconômicos desempenham papel decisivo na manutenção da trajetória escolar dos estudantes do ensino médio. Os resultados da análise sugerem que estratégias baseadas exclusivamente em melhoria de infraestrutura escolar ou ampliação de recursos físicos apresentam impacto limitado diante da complexidade do problema, assim, o enfrentamento da evasão exige ações coordenadas que combinem políticas educacionais, sociais e econômicas; garantindo aos jovens brasileiros não apenas o acesso, como também condições reais de concluírem suas trajetórias escolares.




















%====================================================================


%See the guidelines for metadata and references:
%https://sol.sbc.org.br/journals/index.php/rbie/libraryFiles/downloadPublic/71
%====================================================================

\pagebreak
\pagebreak
%\nocite{*}
%\printbibliography


\end{document}

