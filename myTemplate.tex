%
% Template for RBIE papers in LaTeX
%

% The above language combination is for this template document only.
% You should use one of the following:
\documentclass[english, spanish, brazilian]{RBIEarticle} % for papers in portuguese
%\documentclass[brazilian, spanish, english]{RBIEarticle} % for papers in english
%\documentclass[brazilian, english, spanish]{RBIEarticle} % for papers in spanish

% Papers in Portuguese or Spanish may require the following lines:
\usepackage[utf8]{inputenc} % chooses UTF-8 as the main character set
\usepackage[T1]{fontenc} % for correct syllable separation in accented words

% The next two statements are needed for the example table in this document
% (i.e. you don't necessarily need them in your own paper)
\usepackage{colortbl}
\definecolor{gray}{gray}{.8}

% Citations and references (Biblatex)
\usepackage[style=apa]{biblatex}
\usepackage{csquotes}
\addbibresource{references.bib}

% Here goes the paper main title
\title{Desenvolvimento de Modelo Probabilístico da Evasão do Ensino Médio nas Escolas Públicas Estaduais da Cidade de São Paulo}

% If the manuscript is written in English, then this element must be removed.
\titleinenglish{<If the Manuscript is in Portuguese or Spanish, then Here Comes the Title in English>}

% If the manuscript is written in English, then this element must be removed.
\titleinspanish{<If the Manuscript is in Portuguese, then Here Comes the Title in Spanish>}

% Here goes the paper author information (repeat for two or more authors)
\author{%
	\parbox{3.8cm}{%
		Bruno Alexandre\\
		Universidade de São Paulo\\
		ORCID: \href{https://orcid.org/0000-0000-0000-0000}{0000-0000-0000-0000}\\
		<author1@my-email>
	}
        \hspace{0.3cm}
	\parbox{3.8cm}{%
		Lucas Gurgel do Amaral\\
		Universidade de São Paulo\\
		ORCID: \href{https://orcid.org/0000-0000-0000-0000}{0000-0000-0000-0000}\\
		<author2@my-email>
	}
        \hspace{0.3cm}
        \parbox{3.8cm}{%
		Rafael de França\\
		Universidade de São Paulo\\
		ORCID: \href{https://orcid.org/0000-0000-0000-0000}{0000-0000-0000-0000}\\
		<author1@my-email>
	}
        \hspace{0.3cm}
	\parbox{3.9cm}{\raggedright%
		Richard Pereira do Nascimento\\
		Universidade de São Paulo\\
		ORCID: \href{https://orcid.org/0000-0000-0000-0000}{0000-0000-0000-0000}\\
		<author2@my-email>
	}
}

\Submission{dd/Mmm/yyyy}
\First_round_notif{dd/Mmm/yyyy}
\New_version{dd/Mmm/yyyy}
\Second_round_notif{dd/Mmm/yyyy}
\Camera_ready{dd/Mmm/yyyy}
\Edition_review{dd/Mmm/yyyy}
\Available_online{dd/Mmm/yyyy}
\Published{dd/Mmm/yyyy}

% Here goes the page heading information
\heading{Surname, initials Author 1, Surname, initials Author 2 (for 1 to 2 Authors)	\\
Last author’s surname et al. (for more than 2 authors)
}{RBIE v.VV – yyyy}

% And finally here goes the citation information
\citeas{Last name, Initials., \ldots \& Last name, Initials.  (Year). Article title in the original language. Revista Brasileira de Informática na Educação, vol, pp-pp. https://doi.org/10.5753/rbie.yyyy.id}

%====================================================================
%\hyphenpenalty=10000
%\setcounter{page}{01}

\begin{document}
\maketitle

% If the manuscript is written in English, then this element must be removed.
\begin{otherlanguage}{brazilian}
\begin{abstract}
A evasão escolar no ensino público brasileiro, caracterizado pelo abandono das atividades escolares regulares, vem se mostrando como um grande desafio na qualifacação e alfabetização do público infanto-juvenil, principalmente nas camadas menos favorecidas da sociedade brasileira. 
\end{abstract}
\end{otherlanguage}

\begin{otherlanguage}{brazilian}
\begin{abstract}
<Aqui vem o resumo do artigo em português. O resumo deve resumir o conteúdo do manuscrito e deve conter no mínimo 150 e no máximo 300~palavras e deve ser escrito em itálico, Times~10, justificado, sem recuo especial e sem espaçamento antes ou depois.>
\keywords <O resumo deve ser seguido de 3 a 10 palavras-chave. As palavras-chave devem ser justificadas com espaçamento simples, sem recuo especial, sem espaçamento antes e espaçamento exato de 24 pontos depois. O texto deve ser configurado com a fonte Times em tamanho de fonte 10 e em estilo de fonte itálico. Use ponto e vírgula como separador. As palavras-chave devem iniciar com letra maiúsculas.>
\end{abstract}
\end{otherlanguage}

\begin{otherlanguage}{english}
\begin{abstract}
<Here comes the abstract of the paper in English. The abstract should summarize the contents of the manuscript and should contain at least 150 and at most 300~words long and must be written in italics, Times~10, justified, with no special indentation and no spacing before or after.>
\keywords <Abstract must be followed by 3 to 10 keywords. The keywords should be justified with a line space single, no special indentation, with no spacing before and spacing of exactly 24-points after. The text should be set in Times 10-point font size and in italic font style. Please use semi-colon as a separator. Keywords must be title cased.>
\end{abstract}
\end{otherlanguage}

% If the manuscript is written in English, then this element must be removed.
\begin{otherlanguage}{spanish}
\begin{abstract}
<Aquí viene el resumen del artículo en español. El resumen debe resumir el contenido del manuscrito y debe contener un mínimo de 150 y un máximo de 300~palabras y debe estar escrito en cursiva, Times~10, justificado, sin sangría especial y sin espacio antes o después.>
\keywords <El resumen debe ir seguido de 3 a 10 palabras clave. Las palabras clave deben estar justificadas con un espacio de línea simple, sin sangría especial, sin espacios antes y con un espacio de exactamente 24 puntos después. El texto debe configurarse fuente Times con tamaño de 10 puntos y en estilo de fuente cursiva. Utilice punto y coma como separador. Las palabras clave deben comenzar con una letra mayúscula.>
\end{abstract}
\end{otherlanguage}

\pagebreak

%====================================================================

\section{Introdução}
A evasão escolar é definida como a ausência de retorno ao sistema de ensino formal do aluno em idade escolar após abandono ou reprovação. Deste modo, enquanto o abandono faz referência à situação do aluno que deixa de frequentar as aulas durante o ano letivo, a evasão refere-se ao aluno que, por qualquer motivo, não regressou à rede de ensino com o reinicio do ano letivo, assim como descrito pelo manual guia do Sistema de Alerta Preventivo  de Evasão e Abandono Escolar (SAP). 

Elucidada sua definição, a evasão se mostra um grande desafio na esfera do ensino brasileiro, em especial ao ensino público. Segundo os indicadadores de taxa de transição e fluxo provenientes do censo escolar, a nível nacional e na etapa do ensino médio, enquanto a educação privada tem taxa de evasão de 2\% a 3\% na série histórica, escolas públicas estaduais apresentam evasão de 7\% a 12\% do contingente total de alunos. 

Quando analisado a nível municipal em São Paulo com base na série histórica de 2014 a 2021, também no ensino médio, a diferença é levemente superior. A rede privada apresenta taxa de 1\% a 2,5\% de evasão, ao passo que a rede estadual exibe taxa de 6\% a 12\%. Com base no último registro mensurado, em 2021, 8,7\% de alunos do ensino estadual se evadiram em comparação com 2,6 por cento do ensino privado. Para mais, proporcionalmente há um maior número de matrículas no ensino público em relação ao ensino privado, 340 mil e 81 mil em 2021 respectivamente segundo dados do INEP, o que reforça a gravidade do retorno ao ensino público. 

Acerca das causas que levam à evasão, Silva (2013) ressalta que de modo geral o nível socioeconômico das famílias é um dos principais fatores que impele estudantes a não retomar a vida acadêmica. Um baixo nível socioeconômico, que reflete também a pobreza, induz indivíduos a abandonar os estudos à procura de trabalho para incoporar a renda familiar mensal; da mesma forma destaca-se a dificuldade em conciliar estudo e trabalho, procedente do contexto econômico-social do aluno. Silva (2013) também aponta que não há um conjunto de técnicas e sistemas governamentais para rastrear e entender o abandono e evasão escolar nas redes brasileiras de ensino, o que poderia ser aspecto-chave para a desinformação e, por conseguinte, à continuidade do problema.

Este artigo investiga a problemática da evasão nas escolas estaduais da cidade de São Paulo a partir da Pesquisa Nacional por Amostra de Domicílios (PNAD) com a finalidade de analisar e desenvolver modelos preditivos baseados em aprendizado de máquina para avaliar a probablidade de evasão escolar de alunos com base em suas características socioeconômicas, a fim de proporcionar um método probabilístico para previnir evasão e abandono escolar e compreender como o perfil do estudante impacta seu desenvolvimento acadêmico. 

\section{Links importantes}
https://app.powerbi.com/view?r=eyJrIjoiN2ViNDBjNDEtMTM0OC00ZmFhLWIyZWYtZjI1YjU0NzQzMTJhIiwidCI6IjI2ZjczODk3LWM4YWMtNGIxZS05NzhmLWVhNGMwNzc0MzRiZiJ9

https://www.gov.br/inep/pt-br/acesso-a-informacao/dados-abertos/inep-data/estatisticas-censo-escolar

TEODORO, L. de A.; KAPPEL, M. A. A. Aplicação de Técnicas de Aprendizado de Máquina Para Predição de Risco de Evasão Escolar em Instituições Públicas de Ensino Superior no Brasil. Revista Brasileira de Informática na Educação, [S. l.], v. 28, p. 838–863, 2020. DOI: 10.5753/rbie.2020.28.0.838. Disponível em: https://journals-sol.sbc.org.br/index.php/rbie/article/view/3691. Acesso em: 17 ago. 2025.

OKEWOLE, Dorcas Modupe. A Dummy Variable Regression on Students' Academic Performance. 2012.

FERREIRA, Elen Cristina da Silva; OLIVEIRA, Nayara Maria de. EVASÃO ESCOLAR NO ENSINO MÉDIO: causas e consequências . Scientia Generalis, [S. l.], v. 1, n. 2, p. 39–48, 2020. Disponível em: https://scientiageneralis.com.br/index.php/SG/article/view/v1n2a4. Acesso em: 19 ago. 2025.

RADÜNZ, Angela. Regressão logística. 1992.

\section{Fundamentos Teóricos}

\section{Trabalhos Relacionados}

\section{Metodologia}
A Pesquisa Nacional por Amostra de Domicílios (PNAD), realizada pelo Instituto Brasileiro de Geografia e Estatística (IBGE), é um conjunto de informações detalhadas sobre o cenário socioeconômico da sociedade brasileira. A partir dela, é possível extrair dados e informações das características gerais da população. A metodologia deste artigo consiste no processamento desses dados e a utilização de técnicas de aprendizado de máquina que possibilitem a previsão da probabilidade da evasão do aluno dadas as suas características de cunho social e econômico. Para tal, foram utilizadas as linguagens R, Python e a biblioteca Scikit-learn.

A base de dados foi obtida diretamente a partir do website do IBGE na seção PNAD Contínua e lida preliminarmente em R para interpretação dos dados de largura fixa a largura variável com o auxílio da biblioteca PNADcIBGE. Foram filtradas apenas os resultados pertinentes, a partir do de ano de 2016, à analise de occorência de pessoas que são de São Paulo, não frequentam mais a escola, frequentavam a rede pública de ensino, frequentaram escola alguma vez, cursaram como grau mais elevado o ensino médio regular ou 2º grau, concluiram o curso que frequentaram e com idade a partir de 14 anos. Com base nesses dados é possível rotular ocorrência de evasão ou conclusão do ensino médio como variável binária exclusiva dependente.

Após o procedimento de filtragem, foi feita uma seleção das características, ou colunas, mais relavantes para explicar o fenômeno da evasão escolar. Características como sexo, cor ou raça, se já trabalhou ou estagiou por pelo menos 1 hora em alguma atividade remunerada em dinheiro, se nos últimos três meses, alguma vez, algum morador com menos de 18 anos de idade, sentiu fome, mas não comeu porque não havia dinheiro para comprar comida, número de componentes do domicílio (exclusive as pessoas cuja condição no domicílio era pensionista, empregado doméstico ou parente do empregado doméstico) e rendimento domiciliar per capita
(habitual de todos os trabalhos e efetivo de outras fontes), as quais são alguma das características que mais se correlacionam qualitativamente ao abandono e à evasão escolar no ensino médio (FERREIRA; OLIVEIRA, 2020).

Foi gerado um arquivo CSV com os dados já filtrados e características selecionadas, e posteriormente este foi lido em Python para limpeza de dados faltantes, balanceamento de categorias pela técninca de undersampling, conforme aplicado em Teodoro e Kappel (2020), e verificação dos tipos de variáveis com o objetivo de utilizá-lo como base de treinamento aos modelos de aprendizado de máquina.


\subsection{Pré-processamento dos Dados}
Para simplificação e integridade dos modelos de regressão (OKEWOLE, 2012), as variáveis categóricas foram transformadas em variáveis binárias exclusivas. Variáveis numéricas foram padronizadas de acordo com a média das amostras e seu desvio padrão.

Para o treino e teste efetivo dos modelos, 80\% da base de dados foi destinada ao treinamento e o restante, 20\%, a testes para verificação das métricas. 

\subsection{Técnicas de Aprendizado de Máquina}
Este Trabalho apresenta três técnicas de aprendizado de máquina para efeito de modelagem probabilística, sendo elas Regressão Logística, Redes Neurais e Florestas Aleatórias de Classificação. Tais técnicas foram utilizadas devido a seus retornos baseados em estimativas em valores reais entre 0 e 1, que são consistentes com o modelo de probabilidades. Destaca-se que a escolha de diferentes técnicas reflete unicamente o objetivo de obter o mais acurado desempenho preditivo para o problema de previsão de probabilidades.

A Regressão Logística, conforme Radünz (1992), é um método de regressão compatível com modelagens de variável dependente binária e variáveis independetes numéricas ou categóricas. 

\subsection{Métricas}


\section{Manuscript Preparation}
Before formatting the manuscript, please read carefully the guide for authors, which presents important information on ethics, selection criteria, guidance on methodology and submission instructions. Then, this template may guide the authors to correctly format the manuscript.

The number of pages of the manuscript must be in between 15--30, excluding references and appendices.


\subsection{Page Setup}
The paper size must be set to A4 (210x297 mm). The document margins must be 2.3 cm in all sides (Top: 2.3 cm; Bottom: 2.3 cm; Left: 2.3 cm; Right: 2.3 cm).

Regarding the page layout, authors should set the vertical alignment to the top, and the header and the footer: 1.25 cm.

Any text or material outside the margins might have problems with overlay information.


\subsection{Title}
Use 17-point Times New Roman for the title in the original language, center aligned, line space multiple 1,15 with a bold font style, with an additional spacing of 10-point after, and initial letters capitalized. Articles and words like ``is'', ``or'', ``then'', etc. should not be capitalized unless they are the first word of the title.

If the original language of the manuscript is Portuguese or English, then it is necessary to add the title in English. After the title in the original language, add in the next paragraph a “Title:” heading and the corresponding title in English just in the sequence. Use 12-point Times New Roman type, center aligned, line space multiple 1,15 with an italic font style, with an additional spacing of 24-point after, and initial letters capitalized. As in the title in the original language, articles and words like ``is'', ``or'', ``then'', etc. should not be capitalized unless they are the first word of the title.


\subsection{Authors’ Information}
Use 10-point Times New Roman type for the authors’ information, aligned left, line space single with an italic font style, with no additional spacing. In the first line of authors’ information provide the full name of the authors, in the second line his affiliation and in the last line his email address. If there is only one author, this information must be in the center. If there are two or three authors, then organize the information respectively in two or three columns. For more authors, consider organizing the information in two or three columns and as many lines as necessary.

Notice that all information that may identify the authors must be suppressed of the manuscript at the submission phase. However, this information must be filled in the metadata of the submission form as well as in the final version of the manuscript in case of acceptance.


\subsection{Abstract and Keywords}
Each paper must have an abstract. The abstract should appear justified, with a line space single, no spacing before and after, and font size of 10-point Times in italic. The abstract should summarize the contents of the manuscript and should contain at least 150 and at most 300 words long. The sentence must end with a period. Before the abstract text there is one line with the ``Abstract'' heading in font size of 12-point Times in bold.

In the next line after the abstract, comes the keywords, with the heading ``Keywords:'' in font size of 10-point Times in bold, followed by 3 to 10 keywords aligned left, with a line space single, no special indentation, with no spacing before and 24-points spacing after. The keywords should be set in Times 10-point font size and in italic font style, with 24-point spacing after. To separate the keywords, use the semi-colon. Keywords must be title cased.

If the manuscript is written in Portuguese or Spanish, the authors must provide the abstract and keywords in this language before providing the one in English.


\subsection{The Sections}
The document is organized in only one column. The section text must be set to 12-point Times, justified, line space single and 6-point spacing after.

Section, subsection and sub subsection first paragraph should not have the first line indent, other paragraphs should have a first line indent of 0.75 centimeter.


\subsubsection{Section Titles}
The heading of a section title must be 14-point bold with initial letters capitalized, aligned to the left with a line space sing, with an additional spacing of 24-point before and 12-point after. After the title number, there should be no dot.


\subsubsection{Subsection Titles}
The heading of a subsection title must be 12-point bold with initial letters capitalized, aligned to the left with a line space single, with an additional spacing of 12-point before and 6-point after. After the title number, there should be no dot.


\subsubsection{Sub Subsection Titles}
The heading of a sub subsection title should be in 12-point italic with initial letters capitalized, aligned to the left with a line space single, with an additional spacing of 12-point before and 6-point after. After the title number, there should be no dot.


\subsection{Tables}
Tables (e.g., \autoref{tab:one}) must be positioned preferably at the top or bottom of the page within the given margins. Avoid breaking tables on different pages, unless it does not fit one page only. Tables should be properly numbered, centered and should always have a caption positioned above it. Captions should be centered with 9-point Times, with 12-point spacing before and 6-point after. The final sentence of a caption must end with a period.

Table text should be 10-point Times, with no spacing before or after.

\begin{table}[h]
	\caption{Caption table 1}
	\label{tab:one}
	\centering\footnotesize%
	\begin{tabular}{|c|c|}
		\hline
		\rowcolor{gray} \textbf{Example column 1} & \textbf{Example column 2}\\
		\hline
		Example text 1 & Example text 2\\
		\hline
	\end{tabular}
\end{table}


\subsection{Figures}
Figures should be produced electronically and integrated into the document. As they may lose quality when integrated into the document, it is important to check if it is with a good resolution (at least 300 dpi is recommended). Check line drawings, grids, and details within the figures that must be clearly readable and may not be written one on top of the other, considering 100\% view and print version.

Figures (e.g., \autoref{fig:one}) must appear inside the designated margins. They should be properly numbered, centered and should always have a caption positioned under it. Captions should be centered, with 9-point font size. Spacing before and after should be of 6-point and 12-point, respectively.

The final sentence of a caption must end with a period. 

\begin{figure}[h]
	\centerline{\includegraphics[scale=0.25]{newlogo.png}}
	\caption{Caption figure 1}
	\label{fig:one}
\end{figure}


\subsection{Equations}
Special attention with equations as some characters may be lost as well as formatting. Equations (e.g., \autoref{eq:one}) should be placed on a separate line, numbered and centered. An extra line space should be added below the equation. The numbers accorded to equations must appear enclosed in brackets and positioned right side (with some space after the equation).
The use of a table with two columns is advisable.

\begin{equation}
	a = b + c
	\label{eq:one}
\end{equation}


\subsection{Program Code}
Program listing commands in text (e.g., \autoref{code:one}) should be set in 9-point Courier New, with no spacing before or after, and no first line indent. Codes must appear inside the designated margins, with external borders and they should be properly numbered. Captions should be centered, with 9-point font size. Spacing before and after should be of 6-point and 12-point, respectively.

\begin{code}[h]
	\begin{lstlisting}
begin
    Writeln('Hello World!!');
end.
	\end{lstlisting}
	\caption{Example of code}
	\label{code:one}
\end{code}


\subsection{In-Text Citations and Reference List}

When you use others' ideas in your paper, you should credit them with an in-text citation. In-text citations must follow APA 7 Style, which consist of the surname of the authors and the year of publication. More on \href{https://apastyle.apa.org/}{Writing In-Text Citations in APA Style}, please refer to \href{https://libguides.brenau.edu/APA7}{APA Citation Guide (7th edition)}.

The  \href{https://libguides.brenau.edu/APA7}{APA Citation Guide (7th edition)} explains why and what to cite, citing references in text, the purpose of the reference list and how to build the reference list. It is possible to find more information on  \href{https://libguides.brenau.edu/APA7}{APA Citation Guide (7th edition)} and on how to deal with missing information as well as class notes, class lectures, presentations, social media, among other sources. Some sample references are provided by the  \href{https://libguides.brenau.edu/APA7}{APA Citation Guide (7th edition)}.

The reference list must be ordered alphabetically. References should be set to 12-point, justified, with a single line space, 6-point additional spacing after and hanging indent of 0.75 centimeter.

Citation 1 \parencite{Baker2011}

Citation 2 \parencite{Seffrin2013}

Citation 3 \parencite{Brasil2008}

Citation 4 \parencite{Kautzman2015}

Citation 5 \parencite{Sweller1991}

Citation 6 \parencite{Clark2006}

Citation 7 \parencite{Mason2012}


\section{Conclusions}
We hope you find the information in this template useful, and it helps you in the preparation of your manuscript.

If you find inconsistencies or need additional information, please contact the editors.


\section*{Acknowledgements}
%Place the acknowledgements only in the final version of the manuscript, after acceptance. They should be placed before the references section without numbering.
Our special thanks to Rafael Bohrer Ávila, Matheus Segalotto and Bruno Fagundes da Silva for their help with this latex template. 


%====================================================================

\printbibliography
%See the guidelines for metadata and references:
%https://sol.sbc.org.br/journals/index.php/rbie/libraryFiles/downloadPublic/71
%====================================================================


 \section*{Appendix 1}
\label{apendice1}

If any, the appendix should appear directly after the references without numbering, and not on a new page.

\begin{enumerate}
    \item[A] When the reference has a Link
    \begin{itemize}
        \item Make a clickable link on the respective URL (if you are using MS-Word, use the tool Insert Hyperlink, informing the URL).
    \end{itemize}
    \item[B] Allow readers to search for the reference on Google Scholar
    \begin{itemize}
        \item Copy the title of the reference and put in between ``\%22'', including the ``+'' character between each word: http://scholar.google.com/scholar?q=\%22PASTE+TITLE+\\HERE\%22\&hl=en\&lr=\&btnG=Search
        \item If it is a common title, you may add the author, such as in: \\http://scholar.google.com/ scholar?q=PASTE+AUTHOR+HERE+\%22PASTE+TITLE\\+HERE\%22\&hl=en\&lr=\&btnG=Search
        \item Or you may use the publication year (YEAR) to restrict the results, such as in:\\ http://scholar.google.com/scholar?
        q=PASTE+AUTHOR+HERE+\%22PASTE+TITLE\\+HERE\%22\&hl=en\&lr=\&btnG=Search \&as\_ylo=YEAR\&as\_yhi=YEAR
        \item It is highly advisable to confirm if the link is correct (and if Google Scholar presents a correct result).
        \item Include the term ``[GS SEARCH]'' at the end of each reference and make “GS SEARCH” a hyperlink with the URL just created.
    \end{itemize}
    \item[C] Allow readers to access references with DOI
    \begin{itemize}
        \item Add DOI hyperlink (make it clickable) using the corresponding URL (the URL can be created adding ``http://doi.org/'' in front of the DOI hyperlink).
    \end{itemize}
\end{enumerate}

\end{document}
